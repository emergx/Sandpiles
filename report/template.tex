% article.tex, a sample LaTeX file.
% Run LaTeX on this file twice for proper section numbers.
% A '%' causes LaTeX to ignore remaining text on the line

% Use the following line for draft mode (double spaced, single column)
%\documentclass[preprint,pre,floats,aps,amsmath,amssymb]{revtex4}

% Use the following line for journal mode (single spaced, double column)
\documentclass[twocolumn,pre,floats,aps,amsmath,amssymb]{revtex4}
\usepackage{graphicx}
\usepackage{bm}

\begin{document}

\title{The Scientific Paper: A Template with Tips for Working with
\LaTeX}
\author{D.P. Jackson, K. Browne, and J.Q. Student}
\affiliation{Department of Physics and Astronomy, Dickinson
College, Carlisle, Pennsylvania  17013  USA}
\date{\today}

\begin{abstract}
This paper should serve as an outline for writing scientific papers.
It contains all of the important sections that should be included in
scientific paper as well as descriptions of what should be included in
each of these sections. It also contains some useful tips on how to
use \LaTeX\ to write scientific papers.  The easiest way is to use
this document as a template and insert your text and figures as
described in the text below. This section is the abstract.  The
abstract should contain a brief description of the project including
relevant description of the problem, data collection procedures, and a
summary of results as well as a brief description of how this
information fits into the overall field.  The abstract may contain
equations like $\textbf{E}=\textbf{E}_0\cos
(\textbf{k}\cdot\textbf{r}-\omega t+\phi)$ and, as you will notice,
inline equations look much better in \LaTeX\ than they do in MS Word.
An abstract is usually quite short. Often, the length is limited to
between 200 and 400 words.
\end{abstract}

\maketitle

\section{Introduction}
\label{sec:intro}

This paper contains a general outline of the information that should
be included in a scientific paper.  It provides a good template within
which you can easily write a paper.    When you start out writing
papers, you will likely include most of these sections and utilize
this fairly standard format. As you gain experience, you may choose a
different ordering or different sections as you find appropriate. 
Remember this is just a template to help you get started.  You will
have your own style of writing. Your audience and the content of your
paper should be the most important guiding influence when writing any
paper.  The writing process will go much more smoothly if you take
some time to answer a few questions before you begin writing. For
example, before you begin writing, ask yourself, ``Who is my
audience?'', ``What do I want them to get out of this paper?'', and
``What are the most important ideas to convey with this paper?'' 
There are lots of other questions you could ask, but these three will
help you generate a document that is pitched at the right level and
contains information that is useful to your audience.

You should keep in mind that a good scientific paper always introduces
the reader to the subject material and provides the appropriate
background information that the author thinks the reader will need.  A
good scientific paper will always make the experimental,
computational, or theoretical methods clear enough so that a competent
reader should be able to reproduce the work.  A clear description of
how any data was collected should be included as well as a description
of the raw data, generally in graphical format. Any analysis performed
on the data should be outlined clearly. Analysis and conclusions drawn
from the analysis should generally be described separately from raw
data.  A paper should end with a set of conclusions based on the
analysis.

It is the responsibility of the author to carefully lead the reader
from the experimental design through the conclusions tying each piece
together.  For example, it should be clear to the reader explicitly
how your analysis leads from your raw data to your conclusions.  If
you do not make this clear, no matter whether or not you are right,
you have not done your job as an author and will find that you have a
hard time convincing anyone that what you have done is valid. 
Finally, every paper should end with a references section.  A
scientific paper without any references, indicates that the author
believes that every thought conveyed in the paper is original. Any
information that you obtain from another source should be cited. The
only exception is for material that is considered common knowledge. 
As a student, your common knowledge will often be somewhat more
limited than the average author in a scientific journal.  As such, you
will often reference information from class notes or textbooks that
other authors may not.  When in doubt, make a reference.  This
eliminates any possibility that you will be accused of plagiarism, a
very serious transgression indeed.

An introduction generally contains a brief introduction to the
material that will be presented. Relevant information includes a clear
enunciation of the questions that will be addressed in the paper,
background information relevant for understanding the paper, basic
theory needed to undersand the contents of the paper, etc. 

It is important to take into account your audience when writing the
introduction.  The purpose of an introduction is most often to give
your audience enough information so that they will be able to
understand the rest of your paper and put it into a larger context. 
Depending on your audience, this context may vary.  For example, if
you are preparing a paper with other physics students in mind as the
audience, you will write the introduction so they see how their
previous physics knowledge will be useful in understanding this paper.
 If on the other hand, you are writing this paper for a narrow
selection of researchers, you will not need to include as much
information. Rather, you will present them with enough information so
that they can see how this paper fits in with relevant research.

Because you may not be familiar with \LaTeX, you will undoubtedly have
many questions about how to do certain things.  This document will
serve as a template for producing professional looking papers in
\LaTeX.  Before you begin to modify this document, make sure you have
a copy of it saved somewhere so that you can refer back to it if
needed.  In addition, there are lots of places to get help with
\LaTeX\ (including asking professors in physics and math), but a
useful place to begin is to visit http://www.giss.nasa.gov/latex/. 
All the computers in the physics labs are equipped with a program
called TeXshop that runs the \LaTeX\ engine.

If you have any questions about the appropriate style for a scientific
paper, you should refer to the American Institute of Physics (AIP)
Style Manual at http://www.aip.org/pubservs/style/4thed/toc.html.


\section{Theory}
\label{sec:theory}

Often, if the theory needed to understand a paper is somewhat
extensive, a separate section containing a description of the theory
will be presented.  This section should contain enough theoretical
detail to make it possible for a member of your target audience to be
able to reproduce any results you come up with.  Obviously, the amount
of detail that you include will depend on space constraints and the
expected level of expertise of your audience.  

In the context of a paper written by an undergraduate for a class, you
should include all non-obvious steps and be sure to reference material
that is not ``common knowledge.''  If you just learned the material in
a class, you should include references to where the basic derivation
comes from.  If you start with a non-trivial expression that you had
to look up somewhere, either in a book, a paper, or your notes, you
should definitely include a reference.

All equations should be incorporated into the text using a program
designed to properly format equations. \LaTeX\ is designed to handle
equations, equation numbering, and cross referencing to sections,
equations, and figures with ease.  In fact, you do not need to worry
about numbering any sections or equations, that will be done for you
automatically.  You may want to refer back to an equation, figure, or
section.  To do so, you simply label the appropriate item and then
refer back to it when needed.  For example, to refer back to the
introduction, I can type something like ``this is discussed in
Sections $\backslash$ref\{sec:intro\} and
$\backslash$ref\{sec:theory\}'' to get ``this is discussed in
Sections~\ref{sec:intro} and \ref{sec:theory}.''  Notice that I didn't
have to worry about the sections numbers.  This is a life saver when
you are writing a paper with lots of equations and figures.  Equation
numbering is automatic only in ``displayed math'' mode, which is
illustrated here,
\begin{equation}
\textbf{E}=\textbf{E}_0\cos (\textbf{k}\cdot\textbf{r}-\omega t+\phi),
\label{eq:E}
\end{equation}
and here,
\begin{equation}
\textbf{B}=\textbf{B}_0\cos (\textbf{k}\cdot\textbf{r}-\omega t+\phi).
\label{eq:B}
\end{equation}
Of course, I can easily refer back to Eqs.~\ref{eq:E} and \ref{eq:B}
without having to remember the numbers.

\section{Experimental Methods}
\label{sec:experiment}

This section is often called experimental design or methods.  It
contains information about how you went about your experiment.  The
purpose of this section is to convince your reader that your
experimental methods were sound and thorough. That said, if you have
made experimental errors that you did not correct, or if you made
errors along the way it is your responsibility to report them here. 
If you do not clearly report your experimental methods, you run the
risk of having someone else try your experiment and get other results.
 This then brings into question the validity of your conclusions and
your reputation as a scientist. In addition, if you made errors along
the way that you corrected before collecting your final data, it may
be worth presenting them here so that others can benefit from your
mistakes.

Often you will include a diagram of the experiemtal setup.  This is
shown in Fig.~\ref{fig:geometry} (note that I didn't have to worry
about the figure number).  Of course, \LaTeX\ is a typesetting program
and is not a graphics program, so you will have to make your graphics
in a different program, say, Adobe Illustrator or Xfig.  Fortunately,
including the figures into a \LaTeX document is a pretty simple
matter.

%\begin{figure}[ht]
%\includegraphics[width=2.8 in]{geometry.eps}
%\caption{A sample schematic diagram for an experiment.}
%\label{fig:geometry}
%\end{figure}


Any diagram you include should contain a fairly detailed figure
caption.  A good rule of thumb is that if someone reads the abstract
and looks at all the figures and captions, they should have a
reasonable idea what your paper is about.  While this isn?t always
possible, it is a good thing to shoot for.  That said, this document
doesn?t even come close to meeting that requirement, but it also isn?t
so much a scientific paper as a how to manual on writing one.

As mentioned before, you should include enough information in your
experimental design to make it possible for someone else to reproduce
your experiment.  You should generally outline what you did with
enough detail so that it is clear how you setup your experiment and
how you collected your data.  

It is particularly important to include anything out of the ordinary.
Often we make experimental errors in our setup.  It isn't fun, but it
happens.  If one clearly articulates her setup, it is possible for
others to identify these often subtle experimental errors.

\section{Results}
\label{sec:results}

Your paper should contain a section describing your raw results. 
Often this will be done by including graphs and/or tables of data. 
This data should generally not be heavily processed.  Rather, one
should include results in an understandable format that are a good
representation of the data obtained by your experiment or computation.
 You will have a chance to show processed results in the analysis
section, but in this section you need to present the reader with your
raw data so she can clearly judge the quality of your analysis and
conclusions. 

Often you have far too much data to include it all.  In this case, you
will include a sample of raw data with tables or graphs containing
straightforward compilations of this data. 

It is generally best to make all figures only a single column width,
as shown in Fig.~\ref{fig:force}.  You generally have three choices of
where to place the figures in \LaTeX\.  Here (meaning right here if
possible), top (meaning top of the page if possible), and bottom
(meaning bottom of the page if possible). You may still have to do
some fiddling at the end to get them exactly where you want them.

%\begin{figure}[ht]
%\includegraphics[width=2.8 in]{force.eps}
%\caption{Force as a function of length for a particular experiment.
%The dashed curves represent the nonmagnetic case while the solid
%curves show the magnetic effects.}
%\label{fig:force}
%\end{figure}

There are also times when it is appropriate to include a table of
data.  Unfortunately, tables are not the simplest thing in the world
to do in \LaTeX, but they're not all that difficult either. 
Basically, if you have to make a table, it is best to look for some
help in a book or online and then fiddle until you get it looking the
way you want. Table~\ref{tab:temps} shows an example of a table that
compares two sets of temperature data. As you might expect, simpler
tables are easier to make.

\begin{table}[ht]
\caption{Conventional and syringe thermometer readings. The highest
and lowest readings were used for calibration.}
\begin{center}
\begin{tabular}{@{\hspace{18pt}} c @{\hspace{18pt}} ||
@{\hspace{12pt}} c @{\hspace{12pt}} | @{\hspace{12pt}} c
@{\hspace{12pt}} }

\hline\hline
Conventional & \multicolumn{2}{c}{Syringe {\hspace{9pt}} } \\ \hline
20$^\circ$C & 1.8cc & 20$^\circ$C \\
27$^\circ$C & 2.4cc & 28$^\circ$C \\
42$^\circ$C & 3.9cc & 46$^\circ$C \\
55$^\circ$C & 5.0cc & 59$^\circ$C \\
67$^\circ$C & 6.0cc & 72$^\circ$C \\
84$^\circ$C & 7.0cc & 84$^\circ$C \\
\hline\hline
\end{tabular}
\end{center}
\label{tab:temps}
\end{table}

In general, you should never include a table in a paper when a
figure/graph will do a better job.  It is quite rare to see tables in
scientific papers.  You should never include a long list of data or an
excerpt from a spreadsheet unless the particular values in the list
are very important.  Long lists are hard to read and generally confuse
or bore your reader.  

Most often tables are used to show a few numbers derived from a larger
dataset.  This is a good use of tables but should generally occur in
the analysis sections because the numbers are derived from the data.

Here is another table. We can reference this table in the same way
mentioned in Section 2. Table~\ref{tab:pressure} shows a slightly
simpler table.  

\begin{table}[ht]
\caption{Force, area, and pressure data for the experiment shown in
Fig.~\ref{fig:geometry} and described by Eq.~\ref{eq:B}.  Agreement is
typically within five percent.}
\begin{center}
\begin{tabular}{l @{\hspace{30pt}} c @{\hspace{18pt}} c}
\hline\hline
& Piston 1 & Piston 2 \\ \hline
Avg. Force (N) & 4.40 & 2.25 \\
Area (cm$^2$) & 6.16 & 2.25 \\
$F/A$ (N/cm$^2$) & 0.714 & 0.717 \\
\hline\hline
\end{tabular}
\end{center}
\label{tab:pressure}
\end{table}

\section{Analysis}
\label{sec:analysis}

After you have clearly described your results, you will describe how
you will analyze these results, that is, how you will process the data
you collected to obtain information that will help you answer the
questions you brought up in the introduction. 

It is critically important that the analysis section of a paper is
clear.  Your job in the analysis section is to convince the reader
that the methods you used to get from your results to your conclusions
are sound.  If your analysis section is incomplete or unclear, your
reader may not trust the conclusions you draw.  

This is another section where you will often have equations, graphs
and tables.  Remember that whenever you use an equation, graph or
table, it should be referred to in the text.  Any equation, graph,
figure, or table should fit into your explanations.  If you include a
graph but make not mention of it in the text, the graph either has not
reason to be included, or you have omitted important information from
the text.

\section{Conclusion}
\label{sec:conclusion}

Your conclusions section should be brief, but long enough to refocus
the reader.  The conclusions section describes your assertions based
on your data.  In essence, it contains the answers you?ve come up with
for the questions you asked in the introduction.  

You should also make it a point to place your conclusions within a
context.  That is, you should discuss the possible implications of
your conclusions or how they might be relevant to other researchers. 
This is often hard to do as a student, but not impossible. Some
questions you can keep in mind when  writing this section are.  Why
are these conclusions important?  Who might these results affect? 
What could these results be useful for? 

It is important to keep in mind that you should not overstate your
conclusions.  A common error authors make is to over generalize ones
conclusions.  For example, if I find that a particular type of crystal
behaves non-linearly within certain parameters, it is an
overgeneralization to conclude that all crystals of that structure
will behave the same way.  If the author suspects this to be the case,
she can state her prediction, but should not assert that it is a fact
just because she has a hunch based on her experiment with this one
crystal.  This leads us to a final sections that you may or may not
want to include.

\section{Suggestions for Further Research}
\label{sec:further_research}

This section contains a listing of the directions that the author
thinks it will be possible to extend this research.  It can be a list
of possible future experiments or questions one might ask that are
based on the results of the research presented.  This section gives
the author the opportunity to be somewhat more creative.  That said,
it should be clear in the paper that the statements made in this
sections are suggestions, conjecture and or gut reactions.  It is good
to include this kind of information, because it helps one to refine
her intuition and practice asking interesting scientific questions. 
Often, this kind of information goes in the conclusion or a section
called ``discussion.''

\subsubsection*{Including References}

You must also include a references section in any scientific paper. 
To omit the references section is to almost certainly commit
plagiarism. As mentioned before, you should include references
whenever you have used information from another source.  This might be
a professors notes or a textbook.  As you advance in your studies,
your references will come more and more from journal articles since
these articles generally present more recent results.

In \LaTeX, references are handled very easily in a section called
``thebibliography.''  Thus, you won't actually make a section called
references, you will have something called \textit{thebibliography}. 
All you do is add a ``bibitem'' to \textit{thebibliography} and give
it a label.  Then, whenever you want to refer to it, you use a
$\backslash$cite\{\} command.  The order in which you put the items in
``thebibliography'' is the order of the numbering of those items. 
Therefore, make sure you put these items in the order that they appear
in your paper.  Here is an example of a book citation~\cite{FHD}, an
article citation~\cite{Jackson}, and a comment that might make an
important subtle point but one that would detract from the main
text~\cite{Comment}.

That is basically all the sections that are normally included in a
scientific paper, but there are still some issues that might help you
regarding \LaTeX.  I have put these into an appendix as an example of
how you might use an appendix to put material that is essential to
include but inhibits the flow of the paper.

\begin{acknowledgments}

You should always have a short acknowledgements section.  This is
where you thank people who helped you with the project.  These can be
people that assisted with construction, people you talked with that
gave you good ideas, people you had an email correspondence with,
basically anyone that contributed in some way to the success of the
project.  You would also list fundint agencies in the acknowledgements
section.

\end{acknowledgments}


\appendix*

\section{More \LaTeX\ Information}
\label{sec:latex}

This appendix is here to give you a bit more of an introduction to
\LaTeX.  At this point, it is very short and only includes the most
basic items, but I will expand it in the future.  If there is
something you learned about \LaTeX that was very valuable, please let
me know and I will put it in here.

\subsection{Getting Started}

The first thing you have to do is open TeXShop on one of the macs in
the lab and then open a tex file (this one, for example).  Then you
need to typeset the document.  Depending on the options in TeXShop,
the file will compile and produce a PDF file that you can then read or
print out.

\subsection{Fonts}

In a typical scientific paper, you might want to use \textit{italics}
or \textbf{bold} fonts occasionally.  In \LaTeX, these are
accomplished by using the \verb!\textit{}! and \verb!\textbf{}!
commands.  The text you actually want italicized (or in bold) would be
placed inside the curly braces.

\subsection{Math Mode}

In \LaTeX, you enter math mode by typing \verb!$! and then you leave
math mode by typing another \verb!$!.  Thus, to type an equation, you
place it between two dollar signs.  For example, typing \verb!$F=ma$!
results in $F=ma$.  Greek letters are made by typing a backslash and
the name of the greek letter.  For example,
\verb!$\alpha-\beta+\gamma$!  results in $\alpha-\beta+\gamma$.
Superscripts and subscripts are handled by using \verb!^{}! and
\verb!_{}! in math mode respectively.  For example, typing
\verb!$A_{1}=e^{-x^{2}}$! results in $A_{1}=e^{-x^{2}}$.

So far, all of these examples have been for \textit{inline} equations
that occur right in the paragraph you are typing.  More often, you
will want to put equations on lines all by themselves with an equation
number.  This is called a displayed equation and is accomplished by
using the \textit{equation} environment (an environment in \LaTeX is
something that you begin and end such as the \textit{abstract}
environment that was used to create the abstract of this document). 
Thus, to create a displayed equation that has an equation number, you
type \verb!\begin{equation}!, then your equation (and a label), then
\verb!\end{equation}!.  Here is an example:
\begin{equation}
F_m = -\frac{d{\cal E}_m^{(1)} }{db}.
\label{eq:deriv}
\end{equation}
You will have noticed that to make the derivative in
Eq.~\ref{eq:deriv}, I had to make a fraction.  The fraction command is
\verb!\frac{}{}! (in math mode); the numerator goes in the first set
of curly braces and the denominator goes in the second set of curly
braces.  I also used the command \verb!\label{eq:deriv}! in the
equation environment so that I can refer to it simply by typing
\verb!Eq. \ref{eq:deriv}! to get Eq.~\ref{eq:deriv}.

You may also find the need to write vector equations.  There are
different methods of writing vectors in a scientific paper.  Most
textbooks and scientific papers opt to put vectors in bold:
$\textbf{F}=m\textbf{a}$.  This was accomplished by typing
\verb!$\textbf{F}=m\textbf{a}$!.  Note that the command
\verb!\textbf{}! essentially takes you out of math mode and places a
regular boldface letter in the equation.  This is traditionally how
vectors are written in textbooks with the corresponding magnitudes for
$\textbf{F}$ and $\textbf{a}$ written as $F$ and $a$.  This works fine
except when there is no non-math-mode character to make bold.  An
example of a math-mode character that does not have a non-math-mode
equivalent is $\nabla$, obtained by typing \verb!\nabla!.  If you want
this as a vector operator and you want it to be bold, you must use the
command \verb!\bm! (boldmath) to get $\bm{\nabla}$.  This allows you
to write that in general,
\begin{equation}
\bm{\nabla}\times\textbf{A}\ne\bm{\nabla}\cdot\textbf{A}.
\end{equation}
Note the use of \verb!\times! for $\times$, \verb!\cdot! for $\cdot$
and \verb!\ne! for $\ne$.  You might also be interested to know that
unit vectors can be written using the \verb!\hat{}! command.  For
example, \verb!\$hat{\textbf{r}}$! results in $\hat{\textbf{r}}$ and
\verb!$\hat{\textbf{e}}_{\theta}! results in
$\hat{\textbf{e}}_{\theta}$.

One more quick topic that is sure to be useful is how to break
equations.  It is quite common to have equations that are too long to
fit on a single line.  In these instances, you must break the equation
into multiple lines.  This is done using the \textit{equationarray}
environment: \verb!\begin{eqnarray}!$\cdots$\verb!\end{eqnarray}!.  In
an \textit{equationarray}, each line needs to be separated by
\verb!\\! and items surrounded by ampersands (\verb!&!) will be
aligned on separate lines.  Also, each line will be numbered
separately unless you specify\verb!\nonumber! (which is typically what
you'll want to do).  The following shows an example of a multiline
equation:
\begin{eqnarray}
F_{j} &=& \int d {\cal A}\, \Biggl\{ \frac{3\eta \dot{b}}{b^3}
(R^2 - r^2) + [\Psi_j(R) - \Psi_j(r)] \nonumber \\
&\ &\qquad\qquad + \frac{1}{2}\mu_0 \Bigl[ M_{jr}^2(R) -
M_{jz}^2(r) \Bigr] \Biggr\}.
\label{eq:forcej}
\end{eqnarray}
Notice that I have added some space in Eq.~\ref{eq:forcej} to make it
look a little nicer.  For those that really want to make things look
great, fine tuning math equations with a little spacing here and there
can really make a difference.  In math mode, you can add space with
the following commands: \verb!\,! small space, \verb!\:! medium space,
\verb!\;! large space, and \verb&\!& negative small space.  These can
be quite useful in a number of situations.  For example, compare
\verb!\sqrt{2}x! which gives $\sqrt{2}x$ with \verb!\sqrt{2}\,x! which
gives $\sqrt{2}\,x$, or \verb!\int\int dx dy! which gives $\int\int dx
dy$ with \verb&\int\!\!\int \!dx\,dy& which gives $\int\!\!\int
\!dx\,dy$. The differences are subtle, but for those with a discerning
eye, it is wonderful to have such control over your equations. 
Incidentally, the commands \verb!\quad! and \verb!\qquad! add even
larger and larger amounts of space.

Well, that's all for now.  I think that about covers the basics. 
\LaTeX is an extraordinarily powerful program that is capable of
probably anything you can imagine.  However, it is not always obvious
exactly how to accomplish what you want to do.  Fortunately, most of
the ``basics'' are fairly easy and you should have no problem figuring
them out.  For more advanced techniques, you may want to consult a
book or one of the online manuals (there are lots of them).  Have fun
and let me know if you need any help!



\begin{thebibliography}{99}


\bibitem{FHD}R. E. Rosensweig, {\it Ferrohydrodynamics} (Cambridge
University Press, Cambridge, 1985), and references therein.

\bibitem{Jackson}D. P. Jackson, R. E. Goldstein and A. O. Cebers,
Phys. Rev. E {\bf 50}, 298 (1994).

\bibitem{Comment} Here is an example of a comment that you might need
to include.  This is usually a comment about something very subtle
that might be important to include but generally gets in the way of
the regular text.

\end{thebibliography}

\end{document}             % End of document.

